%% lualatex tenuki30
\nonstopmode


\documentclass[11pt,a4paper]{ltjsarticle}
%% \usepackage[noembed]{luatexja-preset}  %% for draft
\usepackage{listings}
\usepackage{graphicx}
\usepackage{here}
\title{手抜きについて}
\author{手抜きチーム \thanks{鈴木太朗, 玉川直樹}}
\date{2020年2月11日}

\begin{document}

\maketitle

手抜きはCSAプロトコルで対局を行うコンピュータ将棋プログラムです。作者らが将棋のプログラムの仕組みを理解することを目的として開発しています。
リポジトリ:https://github.com/hikaen2/tenuki-d


\section{作者}

\subsubsection*{鈴木太朗}
プログラマ。
自転車とEmacsが好きです。
小学校のとき将棋部でした。
棋力は30級程度です。
Twitter: @hikaen2

\subsubsection*{玉川直樹}
好きなコードはadd9です。
将棋ウォーズ2段。
手抜きの定跡を作りました。
Twitter: @Neakih\_kick



\section{今年の目標}


\subsection{リモート参加}
会場外の計算機で指し手を生成することをリモート参加といいます\footnote{世界コンピュータ将棋選手権 大会ルール 第3章 第12条}。
参加者が会場に来ないことではありません。

通信にはインターネットを使いますが,対戦サーバがインターネットに接続されていないためどうにかしてプロキシする必要があります。


\subsection{GUI}
何かしらの方法で盤面をグラフィカルに見たいわけです。
これはHTMLでテーブルを書いてWebブラウザで見る方法が簡単そうです。
去年はRailsで盤面を生成しました (https://github.com/hikaen2/tui)。
今年は盤面をリアルタイムに更新したいです。

\subsection{時間管理}
いつも1手10秒で指していますが,持ち時間をadaptiveに使いたいです。
そのための方法は分かりません。

\subsection{undoMove}
手抜きは手抜きのため局面をundoMoveするかわりに毎回コピーしています。
そのほうがコードが簡単ですが,局所性が悪そうな気がします。

\subsection{駒背番号制}
手抜きは局面を盤面の1次元配列で表しています。
一方で局面を駒の1次元配列で表す方法がありそうです。
これを駒背番号制と呼ぶことにします。

駒背番号制のほうがKPPTで評価しやすそうです。
ただし持ち駒の扱いと,後手番から見たとき(先後反転したとき)の駒の扱いが難しそうな気がします。

\subsection{置換表をStockfishっぽくする}
現在の手抜きの置換表には指し手だけを格納しています。
ハッシュ値,評価値,スコアも入れたほうが良さそうな気がします。

\subsection{指し手のオーダリング}
現在の手抜きは合法手の並び替えをしていませんが,強そうな手から生成したほうがいいと思います。

\subsection{詰将棋}
df-pnできる気がしません。

\subsection{評価バイナリの自作}
できる気がしません。

\subsection{その他}

\begin{itemize}
  \item 差分評価
  \item aspiration search
  \item futility pruning
  \item ponder
\end{itemize}





%% \section{おまけ}

%% \subsection{D言語について}

%% \subsection{CSAプロトコルについて}
%% 手抜きはCSAプロトコル (http://www2.computer-shogi.org/protocol/) で対局します。
%% ほかに有名なプロトコルとしてUSI (Universal Shogi Interface) プロトコルがあります (http://shogidokoro.starfree.jp/usi.html)。
%% WCSCの場合,対戦サーバがCSAプロトコルですから,USIプロトコルでは対局できません。

%% CSAプロトコルの注意点

%% ・本番用サーバで指し手のパケットを分割すると受理されない
%% WCSC28のときに,CSAプロトコルで指し手を複数パケットに分けて送信したところ,受理されませんでした。
%% shogi-server (http://shogi-server.osdn.jp/) ではこの現象が起きません。

%% ・指し手がエラーでないことが分からない

%% エラーがある手を指すと,サーバは\verb|#ILLEGAL_MOVE|というレスポンスを返します:
%% \begin{verbatim*}
%% +0032FU,T1
%% #ILLEGAL_MOVE
%% #LOSE
%% \end{verbatim}

%% この場合,エラーがあることが分かります。

%% 一方,エラーがない手を指すと,サーバはエラーがなかったことを示すレスポンスを返しません:
%% \begin{verbatim*}
%% +0032FU,T1
%% \end{verbatim}

%% この場合,エラーがなかったことが分かりません。

%% 期待する挙動:
%% エラーがあればエラーがあったことを示すレスポンスが返り,エラーがなければエラーがなかったことを示すレスポンスが返る。






\end{document}
